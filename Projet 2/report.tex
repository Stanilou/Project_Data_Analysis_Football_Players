\documentclass[12pt]{scrartcl}

\usepackage[utf8]{inputenc} 
\usepackage[T1]{fontenc}     
\usepackage[french]{babel} 
\usepackage{lmodern}
\usepackage{microtype}
\usepackage{xcolor}
\usepackage{sectsty}
\usepackage{geometry} 
\usepackage{fancyhdr}  
\usepackage{graphicx}        
\usepackage{amsmath, amssymb} 
\usepackage[hidelinks]{hyperref} 

\hypersetup{pdfnewwindow=true}
\setcounter{secnumdepth}{0}   
\definecolor{mainblue}{RGB}{0, 102, 204}
\sectionfont{\color{blue!60!black}\sffamily\Large\bfseries}  
\geometry{margin=2cm}


%%%%%%%%%%%%%%%%%%%%%%%%%%%%%%%%%%%%%%%%%%%%%%%%%%%%%%%%%%%%%%%%%%%%%%%%%%%%%%%%%%%%%%%
%%%%%%%%%%                       EN-TÊTE ET PIED DE PAGE                     %%%%%%%%%%
%%%%%%%%%%%%%%%%%%%%%%%%%%%%%%%%%%%%%%%%%%%%%%%%%%%%%%%%%%%%%%%%%%%%%%%%%%%%%%%%%%%%%%%

\pagestyle{fancy}
\fancyhf{} 

%%% En-tête de page
\fancyhead[L]{\footnotesize IUT de Lannion - Département Informatique}
\fancyhead[R]{\footnotesize 2025} 

%%% Pied de page
\fancyfoot[L]{\footnotesize © Stanislas ROLLAND - Pierre LECHAT - Tous droits réservés} 
\fancyfoot[R]{\footnotesize \thepage}

\renewcommand{\footrulewidth}{0.4pt}


%%%%%%%%%%%%%%%%%%%%%%%%%%%%%%%%%%%%%%%%%%%%%%%%%%%%%%%%%%%%%%%%%%%%%%%%%%%%%%%%%%%%%%%
%%%%%%%%%%                        PAGE DE PRÉSENTATION                       %%%%%%%%%%
%%%%%%%%%%%%%%%%%%%%%%%%%%%%%%%%%%%%%%%%%%%%%%%%%%%%%%%%%%%%%%%%%%%%%%%%%%%%%%%%%%%%%%%

\title{RAPPORT D'ANALYSE\\DÉCISIONNELLE}
\author{Stanislas ROLLAND - Pierre LECHAT}
\date{Novembre 2025}

\begin{document}

    \maketitle
    \tableofcontents  
    \newpage
 

    %%%%%%%%%%%%%%%%%%%%%%%%%%%%%%%%%%%%%%%%%%%%%%%%%%%%%%%%%%%%%%%%%%%%%%%%%%%%%%%%%%%
    %%%%%%%%%%               CONTEXTE ET QUESTION DE RECHERCHE               %%%%%%%%%%
    %%%%%%%%%%%%%%%%%%%%%%%%%%%%%%%%%%%%%%%%%%%%%%%%%%%%%%%%%%%%%%%%%%%%%%%%%%%%%%%%%%%

    \section{Contexte et question de recherche}
        \subsection{Contexte}
            Dans le cadre de notre projet d’analyse décisionnelle, nous avons choisi de nous concentrer sur le domaine de la santé, et plus particulièrement sur la santé mentale et les effets de la musique sur celle-ci. La musique est fréquemment utilisée comme un outil de relaxation et de gestion du stress, mais son influence exacte sur la santé mentale reste encore en grande partie méconnue. Ce sujet a retenu notre attention car il combine notre passion pour la musique et notre souhait de contribuer à une meilleure compréhension des facteurs qui impactent la santé mentale.\\\\
            Après avoir analysé plusieurs jeux de données, nous avons sélectionné un ensemble provenant de la plateforme Kaggle intitulé \href{https://www.kaggle.com/datasets/catherinerasgaitis/mxmh-survey-results}{\underline{"Music and Mental Health Survey Results"}}. Ce jeu de données regroupe des informations sur les habitudes d’écoute musicale des individus ainsi que des indicateurs liés à leur état mental, tels que les niveaux de stress, d’anxiété et de dépression. Ces données nous ont permis de structurer notre réflexion autour de la problématique suivante : "Comment les préférences musicales peuvent-elles refléter ou influencer notre santé mentale ?"
        
            \subsection{Question de recherche}
            La question centrale de notre analyse est de déterminer dans quelle mesure les préférences musicales des individus peuvent être utilisées pour prédire ou comprendre leur état de santé mentale. Plus précisément, nous cherchons à répondre aux questions suivantes :\\
            
            \begin{itemize}
                \item Quel est l'effet global de la musique sur la santé mentale ?\\
                \item Existe-t-il des genres musicaux spécifiques qui sont plus bénéfiques ou nuisibles pour la santé mentale ?\\
                \item La pratique musicale par les instrumentalites et les compositeurs a-t-elle un impact positif sur leur bien-être mental ?\\
                \item Existe-t-il des différences significatives dans l'impact de la musique sur la santé mentale en fonction de variables démographiques telles que l'âge ou le temps d'écoute quotidien ?
            \end{itemize}


    %%%%%%%%%%%%%%%%%%%%%%%%%%%%%%%%%%%%%%%%%%%%%%%%%%%%%%%%%%%%%%%%%%%%%%%%%%%%%%%%%%%
    %%%%%%%%%%               CHOIX DU MODÈLE ET JUSTIFICATION                %%%%%%%%%%
    %%%%%%%%%%%%%%%%%%%%%%%%%%%%%%%%%%%%%%%%%%%%%%%%%%%%%%%%%%%%%%%%%%%%%%%%%%%%%%%%%%%

    \section{Choix du modèle et justification}
        Dans le cadre de ce projet d’analyse décisionnelle, nous avons choisi d’utiliser un modèle appartenant à la catégorie "Probabilités / Risque", à savoir le test du Chi² d’indépendance. Ce modèle statistique est particulièrement adapté à notre problématique, car il permet de déterminer si deux variables qualitatives sont statistiquement liées. Pour notre étude, il permet de vérifier s’il existe une association entre les préférences musicales des individus (leur genre musical favori) et les effets de la musique sur leur santé mentale (une amélioration, une absence d’effet ou une aggravation).\\\\
        En complément de ce modèle décisionnel, nous avons également mobilisé différents outils de statistiques descriptives tels que l’analyse de tendance, les distributions, les comparaisons de moyennes ou encore les corrélations. Ces éléments nous permettent de comprendre la structure globale du jeu de données, d’identifier des tendances majeures et de préparer l’interprétation du modèle principal.


    %%%%%%%%%%%%%%%%%%%%%%%%%%%%%%%%%%%%%%%%%%%%%%%%%%%%%%%%%%%%%%%%%%%%%%%%%%%%%%%%%%%
    %%%%%%%%%%             MÉTHODOLOGIE ET TRAITEMENT DES DONNÉES            %%%%%%%%%%
    %%%%%%%%%%%%%%%%%%%%%%%%%%%%%%%%%%%%%%%%%%%%%%%%%%%%%%%%%%%%%%%%%%%%%%%%%%%%%%%%%%%

    \section{Méthodologie et traitement des données}
        \subsection{Méthodologie abordée}
            Notre méthodologie s'articule autour de plusieurs étapes clés. Nous avons commencé par analyser les liens entre les différentes habitudes d'écoute (genre favori, temps d'écoute, effets ressentis, pratique instrumentale etc) et différents indicateurs de santé mentale (niveaux de stress, d'anxiété et de dépression). Pour chaque paire de variables qualitatives, nous avons appliqué le test du Chi² d’indépendance afin de déterminer s'il existait une relation statistiquement significative entre elles.

        \subsection{Traitement des données}
            Avant de pouvoir effectuer les analyses, nous avons dû procéder à un nettoyage des données. En effet, nous avons supprimé plusieurs colonnes car elles ne contribuaient pas à l’objectif analytique (par exemple "Timestamps", "Permissions" ou "Primary streaming service"). L’absence d’information étant également un problème pour la réalisation de nos analyses, nous avons retiré toutes les lignes contenant des valeurs manquantes. Nous avons également ajouté une colonne "Id" pour identifier chaque individu de manière unique et donc remplacer la colonne "Timestamps".\\\\
            Dans un second temps, plusieurs transformations ont été réalisées pour faciliter l'analyse comme la création de nouveaux attributs tels que les groupes d’âge, la normalisation de tableaux croisés etc.


    %%%%%%%%%%%%%%%%%%%%%%%%%%%%%%%%%%%%%%%%%%%%%%%%%%%%%%%%%%%%%%%%%%%%%%%%%%%%%%%%%%%
    %%%%%%%%%%                   RÉSULTATS ET INTERPRÉTATION                 %%%%%%%%%%
    %%%%%%%%%%%%%%%%%%%%%%%%%%%%%%%%%%%%%%%%%%%%%%%%%%%%%%%%%%%%%%%%%%%%%%%%%%%%%%%%%%%

    \section{Résultats et interprétation}
        %<EXPLICATION_RESULTATS>

    Avec nos graphiques nous avons pu observer beaucoup de choses différentes :
    - Dans la grande majorité des cas la musique (75\%) a un effet positif sur la santé mentale des auditeurs.
    - Dans des très rares cas (2,4\%) la musique a un effet négatif sur la santé mentale des auditeurs.
    - Les genres musicaux les plus bénéfiques sont le gospel et le lofi. Suivi de près par le hip-hop et la country.
    - Les heures d'écoutes quotidiennes n'ont pas d'impact significatif sur l'effet de la musique.
    - On voit qu'en terme de troubles mentaux, les plus atténués par la musique sont l'anxiété et la dépression.
    - Mais que les plus exacerbés par la musique est également la dépression et l'anxiété. 
    - On peut déterminer que plus on est jeune plus la musique a un effet bénéfique sur la santé mentale, mais elle reste utile dans minimum 60\% des cas dans tout les âges.
    - Si jamais une personne est compositrice ou pratique un instrument la musique est plus susceptible d'avoir un effet bénéfique sur sa santé mentale.
    
    %%%%%%%%%%%%%%%%%%%%%%%%%%%%%%%%%%%%%%%%%%%%%%%%%%%%%%%%%%%%%%%%%%%%%%%%%%%%%%%%%%%
    %%%%%%%%%%                LIMITES ET PISTES D'AMÉLIORATION               %%%%%%%%%%
    %%%%%%%%%%%%%%%%%%%%%%%%%%%%%%%%%%%%%%%%%%%%%%%%%%%%%%%%%%%%%%%%%%%%%%%%%%%%%%%%%%%
        
    \section{Limites et pistes d'amélioration}
        \subsection{Limites}
            Plusieurs limites doivent être soulignées, dans un premier lieu, le jeu de données provient d’un questionnaire auto-déclaratif ce qui signifie que les réponses peuvent être "biaisées" par la subjectivité ou la perception personnelle. Les variables sont également majoritairement qualitatives ce qui limite l’utilisation de modèles prédictifs ou d’analyses plus avancées.\\\\
            De plus, le test du Chi² ne permet d’étudier que la dépendance entre deux variables qualitatives. Il n’indique pas la force de la relation ni sa nature. Les données, quant à elles, ne permettent pas d’établir un lien de causalité entre les variables, mais uniquement des associations.

        \subsection{Piste d'amélioration}
            Plusieurs améliorations pourraient être envisagées, l'utilisation d’un jeu de données plus large et plus varié permettrait d’accroître la fiabilité des analyses. Il serait également intéressant de compléter l’étude avec l'utilisation de modèles prédictifs (régression logistique, arbres de décision etc) ou des méthodes de clustering afin d’identifier des profils parmis les auditeurs.\\\\
            On pourrait également envisager une approche temporelle basée sur la mesure de l’évolution de l’état mental en fonction de l’exposition musicale. Cela pourrait apporter une perspective plus dynamique sur l’impact de la musique.
            

    %%%%%%%%%%%%%%%%%%%%%%%%%%%%%%%%%%%%%%%%%%%%%%%%%%%%%%%%%%%%%%%%%%%%%%%%%%%%%%%%%%%
    %%%%%%%%%%                            CONCLUSION                         %%%%%%%%%%
    %%%%%%%%%%%%%%%%%%%%%%%%%%%%%%%%%%%%%%%%%%%%%%%%%%%%%%%%%%%%%%%%%%%%%%%%%%%%%%%%%%%

    \section{Conclusion}
        %<CONCLUSION>


\end{document}